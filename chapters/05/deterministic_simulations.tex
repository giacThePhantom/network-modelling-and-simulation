\graphicspath{{chapters/05/images/}}
\chapter{Deterministic simulations}

\section{introduction}
When the last noise term of equation:

$$X(t+\tau)\approx \vec{x}+\sum\limits_{j=1}^Ma_j(\vec{x})\vec{v}_j\tau + \sum\limits_{j=1}^M\sqrt{a_j(\vec{x})\tau}norm(0,1)\vec{v}_j$$

Becomes negligibly small compared with the second one, a deterministic way to computing the dynamics of a system can be applied.
This happens in the limiting case $a_j(\vec{x})\tau\rightarrow\infty$ and the deterministic simulation produces an average behaviour of the system close to the one obtained by averaging an infinite number of stochastic simulations of the system starting from the same initial state.
Deterministic simulation are faster than exact stochastic ones because reaction events are executed as the simultaneous application of a set of reactions.
A single run is sufficient because stochasticity of the system is not considered any more.
One way to simulate a biological system according to a deterministic approach is to translate it into a set of ordinary differential equations or ODEs.

\section{From biochemical reactions to ODEs}
Ordinary differential equations can be used to simulate a biochemical system that satisfied the spatial homogeneity and the continuum hypothesis.

  \subsection{Starting hypothesis}

    \subsubsection{Continuum hypothesis}
    A biochemical system satisfies the continuum hypothesis if the number of molecules for each species is large enough to safely approximate molecular abundances by concentrations that vary continuously.

    \subsubsection{Effect of starting hypothesis}
    Spatial homogeneity allows to randomize spatial information: the rate of each reaction is space independent.
    The continuum hypothesis allows to approximate discrete changes in molecule number by continuous changes in concentration, moreover individual reactions are infinitesimal changes in molecule abundance.
    This holds for species with molecules counts of thousands or more.
    In the cases where molecule abundance is low, its changes will have to be treated like discrete steps in population size and stochastic simulation should be preferred.

  \subsection{Law of mass action}
  When the two hypothesis are satisfied, a biochemical reaction system can be translated into a set of ODEs by the law of mass action.
  The law of mass action states that the deterministic rate of a chemical reaction is proportional to the product of the concentrations of its reactants.
  Now let $[A] = \frac{\# A}{N_AV}$ be the molar concentration of species $A$ in a chemical volume $V$ and $N_A$ Avogadro's number.
  Some example of conversion of chemical reactions into ODEs are outlined in table \ref{tab:chem-odes}.

  \begin{table}[H]
    \centering
    \begin{tabular}{c c c c}
      \hline
      Reaction type & Reaction & Rate & ODEs\\
      \hline
      Zero-order reaction & $\emptyset\xrightarrow[]{k} A$ & $k$ & $\frac{d[A]}{dt} = k$\\
      First-order reaction & $A\xrightarrow[]{k} B$ & $k[A]$ & $\frac{d[A]}{dt} = -k[A];\frac{d[B]}{dt} = k[A]$\\
      Second-order reaction & $A+B\xrightarrow[]{k} C$ & $k[A][B]$ & $\frac{d[A]}{dt} = \frac{d[B]}{dt} = -k[A][B];\frac{d[C]}{dt} = k[A][B]$\\
      \makecell{Second-order reaction\\ same reactant} & $A+A\xrightarrow[]{k} B$ & $k[A]^2$ & $\frac{d[A]}{dt} = -k[A]^2;\frac{d[B]}{dt} = k[A]^2$\\
      Third-order reaction & $A+B+C\xrightarrow[]{k} D$ & $k[A][B][C]$ & \makecell{$\frac{d[A]}{dt} = \frac{d[B]}{dt} = \frac{d[C]}{dt} =  -k[A][B][C]$\\$\frac{d[D]}{dt} = k[A][B][C]$}\\
      \hline
    \end{tabular}
    \caption{Conversion of biochemical reactions into ODEs}
    \label{tab:chem-odes}
  \end{table}

  Taking into consideration that the deterministic rate $k$ is not the stochastic reaction rate constant $c_j$ but will be computed as in table \ref{tab:chem-rates}.

  \begin{table}[H]
    \centering
    \begin{tabular}{c c c c}
      \hline
      Reaction type & Reaction & Rate & Unit\\
      \hline
      Zero-order reaction & $\emptyset\xrightarrow[]{k} A$ & $k = \frac{c}{N_AV}$ & $concentration\cdot time^{-1}$\\
      First-order reaction & $A\xrightarrow[]{k} B$ &  $k = c$ & $time^{-1}$\\
      Second-order reaction & $A+B\xrightarrow[]{k} C$ & $k = cN_AV$ & $concentration^1time^{-1}$\\
      \makecell{Second-order reaction\\ same reactant} & $A+A\xrightarrow[]{k} B$ & $k = \frac{cN_AV}{2}$ & $concentration^1time^{-1}$\\
      Third-order reaction & $A+B+C\xrightarrow[]{k} D$ & $k = c(N_AV)^2$ & $concentration^2time^{-1}$\\
      \hline
    \end{tabular}
    \caption{Conversion of biochemical reactions into ODEs}
    \label{tab:chem-rates}
  \end{table}

  \subsection{Building the set of ODEs}
  Consider now a biochemical reaction system with $N$ species $S_1, \dots, S_N$ interacting trough $M$ reactions $R_1, \dots R_M$ and a stoichiometric matrix:

  $$\vec{v} = \vec{v}^+-\vec{v}^-$$

  The deterministic rate constant of each reaction is:

  $$k_j = \frac{c_j(N_AV)^{Order_j-1}}{\prod\limits_{i=1}^N\vec{v_{ij}^-!}}$$

  Where $Order_j$ is the order of reaction $R_j$.
  The set of ODEs modelling the evolution of the species is:

  $$\frac{d[S_i]}{dt} = \sum\limits_{j=1}^M\left(k_j\vec{v}_{ji}\prod\limits_{l=1}^N[S_l]^{\vec{v}_{jl}}\right), \forall i = 1, \dots, N$$

  Other methods can be used to approach this translation like the Michaelis-Menten kinetics or the Hill kinetics.
  The former can be used to quantify cooperative binding, the phenomena of binding of a ligand to a macromolecule enhancing when another molecule is attached to the same macro-one.

  \subsection{Michaelis-Menten kinetics}
  Michaelis-Menten (MM) kinetics are used to model enzymatic reactions:

  $$A\xrightarrow[]{E} B$$

  Enzyme accelerate the reaction, and to transform this reaction into a law of mass action it needs to be expanded into:

  $$A+E\xleftrightharpoons[k_2]{k_1} AE\xrightarrow[]{k_3} B+ E$$

  Describing the process more accurately.
  The translation to a set of ODE of the reaction requires the definition of four differential equation, where also $[E]$ and $[EA]$ are considered.
  The MM kinetics allow to simplify the model by reducing the number of equation, so the equation is translated into two odes considering the variation of concentration of $A$ and $B$:

  $$\frac{d[A]}{dt} = -\frac{d[B]}{dt} = -V_{MAX}\frac{[A]}{K_M + [A]}$$

  Where:

  \begin{multicols}{2}
    \begin{itemize}
      \item $V_{MAX}$ is the maximum velocity of the enzymatic reaction.
      \item $K_M$, Michaelis constant, is the concentration of the substrate at which the reaction rate is half of its maximum.
    \end{itemize}
  \end{multicols}

  The effect of the enzyme is modelled, where:

  $$K_M = \frac{k_2+k_3}{k_1}\qquad\land\qquad V_{MAX} = k_{cat}[E_T]$$

  Where $[E_T]$ is the enzyme available to the system.
  This kinetics can be used also in the context of stochastic simulation, so that the propensity for this type of reactions will be:

  $$a(\vec{x}) = \frac{V_{MAX}A}{K_M+A}$$

  Where $V_{MAX}$ and $K_M$ are scaled to consider molecule abundances.

\section{Numerical solution of ODEs}
The simulation of a system of ODE is addressed by solving the initial value or Cauchy problem.
This corresponds to finding the solution of a set of differential equations that satisfies the initial condition corresponding to the initial concentration of the species.
An exact solution is usually to complex, so suitable numerical methods need to be used.
This will produce approximations of the solution at specified time points.
Some interpolation methods can be used to obtain intermediate values.
Even when an exact solution if found, the dynamics of the biochemical system is an approximation.

  \subsection{Finding a solution}
  Consider the system with $N$ species:

  $$\frac{d[X]}{dt} = \vec{F}(t,[X])$$

  Where:

  \begin{multicols}{2}
    \begin{itemize}
      \item $\vec{F}:\mathbb{R}\times\mathbb{R}^N\rightarrow\mathbb{R}^N$ is the vector of $N$ functions providing the time derivatives of species concentration.
      \item $[X]$ is the current state of the system expressed in molecular concentrations.
    \end{itemize}
  \end{multicols}

  Let $I = (0,T_{\max})$ be the integration interval of the system and:

  $$t_n = nh, h>0\land n = 0, \dots, N_h$$

  Be the sequence of discretization of $I$ into subintervals $I_n = [t_n, t_{n+1}]$, where:

  \begin{multicols}{2}
    \begin{itemize}
      \item $N_h$ is the maximum integer such that $t_{N_h}\le T_{max}$.
      \item $h$ is the discretization stepsize.
    \end{itemize}
  \end{multicols}

  Numerical methods compute a sequence of states $[X_n]$.
  Approximating the trajectory in terms of concentrations along the time steps starting from $[X_0]$.
  These methods can be explicit or implicit.
  They are called explicit if $[X_{n+1}]$ can be computed directly from the previous state $[X_k]$.
  They are called implicit if $[X_{n+1}]$ depends implicitly on itself through $\vec{F}$.

  \subsection{Forward/Backward Euler method}
  The Forward Euler method is an explicit method, while the backward one is an implicit.
  The are:

  \begin{align*}
    \text{Forward Euler: } [X_{n+1}] &= [X_n] + h\vec{F}(t_n, [X_n])\\
    \text{Backward Euler: } [X_{n+1}] &= [X_n] + h\vec{F}(t_{n+1}, [X_{n+1}])\\
  \end{align*}

  Comparing this with the chemical Langevin equation, when stochasticity is negligible, the CLE reduces a forward Euler with $\tau=h$.

    \subsubsection{Forward Euler algorithm}
    An implementation of the forward Euler algorithm is found in algorithm \ref{algo:forward-euler}.

    \begin{algorithm}[H]
\DontPrintSemicolon
\SetKwComment{comment}{$\%$}{}
\SetKw{Int}{int}
\SetKw{To}{to}
\SetKw{Return}{return}
\SetKw{Not}{not}
\SetKw{Input}{Input}
\SetKw{Output}{Output}
\SetKw{False}{false}
\SetKw{True}{true}
\SetKwData{Item}{item}
\SetKwFunction{Min}{min}
\SetKwFunction{TitleFunction}{Forward Euler method}

\caption{\protect\TitleFunction{}}
\label{algo:forward-euler}

\Input: a system of ODEs $\frac{d[X]}{dt} = \vec{F}(t, [X])$ corresponding to a biochemical reaction system, the initial state $[X_0]$ of the system with species concentrations at time $0$, the simulation ending time $T_{\max}$ and the discretization stepsize $h$\;

\Output: a trajectory of the biochemical system expressed in terms of molecule concentrations with discretization stepsize $h$\;

$t = 0$\;
$[X] = [X_0]$\;


\While{$t<T_{\max}$}{
	$[X] = [X] + h\cdot\vec{f}(t, [X])$\;
	$t = t+h$\;
}
\end{algorithm}


    \subsubsection{Backward Euler algorithm}
    An implementation of the backward Euler algorithm is found in algorithm \ref{algo:backward-euler}.

    \begin{algorithm}[H]
\DontPrintSemicolon
\SetKwComment{comment}{$\%$}{}
\SetKw{Int}{int}
\SetKw{To}{to}
\SetKw{Return}{return}
\SetKw{Not}{not}
\SetKw{Input}{Input}
\SetKw{Output}{Output}
\SetKw{False}{false}
\SetKw{True}{true}
\SetKwData{Item}{item}
\SetKwFunction{Min}{min}
\SetKwFunction{TitleFunction}{Backward Euler method}

\caption{\protect\TitleFunction{}}
\label{algo:backward-euler}

\Input: a system of ODEs $\frac{d[X]}{dt} = \vec{F}(t, [X])$ corresponding to a biochemical reaction system, the initial state $[X_0]$ of the system with species concentrations at time $0$, the simulation ending time $T_{\max}$ and the discretization stepsize $h$\;

\Output: a trajectory of the biochemical system expressed in terms of molecule concentrations with discretization stepsize $h$\;

$t = 0$\;
$[X] = [X_0]$\;


\While{$t<T_{\max}$}{
	estimate $[X_{new}] = [X] + h\cdot\vec{F}(t, [X])$\;
	$t = t+h$\;
	$[X] = [X] + h\cdot\vec{f}(t, [X_{new}])$\;
}
\end{algorithm}


    \subsubsection{Discussion}
    Implicit methods are less intuitive because they need to compute an estimation of the state.
    This can be done by computing a first approximation of the next state which is used then to compute the actual one.



\section{Improving the accuracy of numerical methods}
The accuracy of the computation depends on the discretization stepsize and on the properties of the numerical method.
The general form of one step for an explicit method is:

$$[X_{n+1}] = [X_n] +h\mathbb{F}(t_n, [X_n], \vec{F}(t_n, [X_n])lh) + h\epsilon_{n+1}(h)$$

Where:

\begin{multicols}{2}
  \begin{itemize}
    \item $n = 0, \dots, N_h$.
    \item $h>0$.
    \item $\mathbb{F}$ is the increment function.
    \item $\epsilon_{n+1}(h)$ is the local truncation error LTE at $t_{n+1}$ of the numerical method.
      This provides a measure of how distant the estimation is from the exact value.
  \end{itemize}
\end{multicols}

A global truncation error is required to evaluate the accuracy of a numerical method.

  \subsection{Global truncation error}
  Consider a numerical method with local truncation error $\epsilon_{n+1}(h)$.
  The global truncation error is:

  $$\epsilon(h) = \max|\epsilon_{n+1}(h)|, n = 0, \dots, N_h$$

  \subsection{Consistency of a numerical method}
  A numerical method with global truncation error $\epsilon(h)$ is consistent with the initial value problem if:

  $$\lim\limits_{h\rightarrow 0}\epsilon(h) = 0$$

  From now on only consistent numerical methods will be considered.

  \subsection{Order of a numerical method}
  A numerical method with global truncation error $\epsilon(h)$ has order $p$ if:

  $$\forall t\in]0, T_{\max}[: \epsilon(h) = O(h^p), h\rightarrow 0$$

  A Taylor expansion shows that the forward Euler has order $1$.
  To increase the accuracy of a simulation the discretization stepsize or increase the order of the numerical methods need to be decreased.
  Decreasing the discretization stepsize implies to compute more simulation steps, while increasing the method order the complexity of each step increases.

  \subsection{Heun method}
  An example of a second order numerical method is the implicit trapezoidal or Crank-Nicolson method, which updates the system by:

  $$[X_{n+1}] = [X_n] + \frac{h}{2}[\vec{F}(t_n, [X_n]) + \vec{F}(t_{n+1}, [X_{n+1}])]$$

  The gain in accuracy is balanced by the increased complexity in the update formula, which requires the evaluation of two $\vec{F}$ at each step.
  This can be transformed into the explicit alternative Heun method, which updates the system by:

  $$[X_{n+1}] = [X_n] + \frac{h}{2}[\vec{F}(t_n, [X_n]) + \vec{F}(t_{n+1}, [X_{n}]+h\vec{F}(t_n, [X_n]))]$$

    \subsubsection{Algorithm}
    An implementation of the Heun algorithm can be found in algorithm \ref{algo:heun}.

    \begin{algorithm}[H]
\DontPrintSemicolon
\SetKwComment{comment}{$\%$}{}
\SetKw{Int}{int}
\SetKw{To}{to}
\SetKw{Return}{return}
\SetKw{Not}{not}
\SetKw{Input}{Input}
\SetKw{Output}{Output}
\SetKw{False}{false}
\SetKw{True}{true}
\SetKwData{Item}{item}
\SetKwFunction{Min}{min}
\SetKwFunction{TitleFunction}{Heun method}

\caption{\protect\TitleFunction{}}
\label{algo:heun}

\Input: a system of ODEs $\frac{d[X]}{dt} = \vec{F}(t, [X])$ corresponding to a biochemical reaction system, the initial state $[X_0]$ of the system with species concentrations at time $0$, the simulation ending time $T_{\max}$ and the discretization stepsize $h$\;

\Output: a trajectory of the biochemical system expressed in terms of molecule concentrations with discretization stepsize $h$\;

$t = 0$\;
$[X] = [X_0]$\;


\While{$t<T_{\max}$}{
  $[X] = [X] + \frac{h}{2}[\vec{F}(t, [X]) + \vec{F}(t, [X]+h\vec{F}(t, [X]))]$\;
	$t = t+h$\;
}
\end{algorithm}


  \subsection{Runge-Kutta methods}
  The Runge-Kutta methods are a family of numerical methods that can be written as:

  $$[X_{n+1}] = [X_n] + h\mathbb{F}(t_n, [X_n], h;\vec{F})$$

  Where:

  \begin{multicols}{2}
    \begin{itemize}
      \item $n = 0, \dots, N_h$.
      \item $h>0$.
      \item $\mathbb{F}$ is the increment function of the method.
    \end{itemize}
  \end{multicols}

  In particular:

  $$\mathbb{F}(t_n, [X_n], h;\vec{F}) = \sum\limits_{i=1}^s b_iK_i$$

  $$K_i = \vec{F}\left(t_n + c_ih, [X_n] + h\sum\limits_{j=1}^sa_{ij}K_j\right)$$

  Whit $s$ being the number of stages of the method and $a_{ij}$, $b_i$ and $c_i$ are suitable numbers that characterize the RK method.
  These method can be explicit or implicit depending on the values of $a_{ij}$.
  The Heun method is an explicit RK method, because $a_{12}$ and $a_{22}$ are zeros.
  The number of stages and the order of the methods are related: the minimum number $s_{\min}$ required to get an explicit RK method of corresponding order is described in table \ref{tab:rk-order}.

  \begin{table}[H]
    \centering
    \begin{tabular}{c | c c c c c c c c }
      Order & $1$ & $2$ & $3$ & $4$ & $5$ & $6$ & $7$ & $8$\\
      \hline
      $s_{min}$ & $1$ & $2$ & $3$ & $4$ & $6$ & $7$ & $9$ & $11$\\
    \end{tabular}
    \caption{RK order and $s_{min}$ relationship}
    \label{tab:rk-order}
  \end{table}

  $4$ is the maximum number of stages for which the order is not less than $s_{\min}$.
  For this a four-stage explicit RK method is the more convenient way to solve an initial-value problem.

    \subsubsection{Fourth order RK method}
    An example of an update of a fourth order RK method is:

    $$[X_{n+1}] = [X_n] +\frac{h}{6}(K_1+2K_2+2K_3+k_4)$$

    Where;

    \begin{multicols}{2}
      \begin{itemize}
        \item $K_1 = \vec{F}(t_n, [X_n])$.
        \item $K_2 = \vec{F}(t_n+ \frac{h}{2}, [X_n] + \frac{h}{2}K_1)$.
        \item $K_3 = \vec{F}(t_n +\frac{h}{2}, [X_n] + \frac{h}{2}K_2)$.
        \item $K_4 = \vec{F}(t_{n+1}, [X_n] + hK_3)$.
      \end{itemize}
    \end{multicols}

    This is called RK4 and is one of the most used numerical methods for deterministic simulations.

      \paragraph{Algorithm}
      An implementation of the RK4 method can be found at algorithm \ref{algo:rk4}.

      
\begin{algorithm}[H]
\DontPrintSemicolon
\SetKwComment{comment}{$\%$}{}
\SetKw{Int}{int}
\SetKw{To}{to}
\SetKw{Return}{return}
\SetKw{Not}{not}
\SetKw{Input}{Input}
\SetKw{Output}{Output}
\SetKw{False}{false}
\SetKw{True}{true}
\SetKwData{Item}{item}
\SetKwFunction{Min}{min}
\SetKwFunction{TitleFunction}{RK4 method}

\caption{\protect\TitleFunction{}}
\label{algo:rk4}

\Input: a system of ODEs $\frac{d[X]}{dt} = \vec{F}(t, [X])$ corresponding to a biochemical reaction system, the initial state $[X_0]$ of the system with species concentrations at time $0$, the simulation ending time $T_{\max}$ and the discretization stepsize $h$\;

\Output: a trajectory of the biochemical system expressed in terms of molecule concentrations with discretization stepsize $h$\;

$t = 0$\;
$[X] = [X_0]$\;


\While{$t<T_{\max}$}{
  $K_1 = \vec{F}(t, [X])$\;
  $K_2 = \vec{F}(t+ \frac{h}{2}, [X] + \frac{h}{2}K_1)$\;
  $K_3 = \vec{F}(t +\frac{h}{2}, [X] + \frac{h}{2}K_2)$\;
  $K_4 = \vec{F}(t+h, [X] + hK_3)$\;
  $[X] = [X] +\frac{h}{6}(K_1+2K_2+2K_3+k_4)$\;
}
\end{algorithm}


\section{Multistep methods}
A numerical method for the approximation of the initial-value problem is a one step method if $\forall n\ge 0$, the computation of $[X_{n+1}]$ depends only on $[X_n]$, otherwise the scheme is called a multistep method.
Multistep (MS) schemes require one functional evaluation at each step and their accuracy can be increased at the expense of increasing the number of steps.
Thy can be implicit or implicit and have an order of accuracy.

  \subsection{Linear multistep numerical method}
  A linear $(s+1)$-step method is a multistep method whose update formula fits the scheme:

  $$[X_{n+1}] = \sum\limits_{j=0}^sa_j[X_{n-j}] + h\sum\limits_{j=0}^sb_j\vec{F}(t_{n-j}, [X_{n-j}]) + hb_{-1}\vec{F}(t_{n+1}, [X_{n+1}])$$

  Where:

  \begin{multicols}{2}
    \begin{itemize}
      \item $n\ge s\ge 0$.
      \item $a_j, b_j$ are numbers that characterize the method.
        When $b_{-1}= 0$ the method is explicit, otherwise implicit.
    \end{itemize}
  \end{multicols}

    \subsubsection{Midpoint method}
    The midpoint method is a second order, two step linear explicit method, which updates the system by:

    $$[X_{n+1}] = [X_{n-1}] + 2h\vec{F}(t_n, [X_n])$$

    They rely on the fact that the formula depend on some previous state of the system to increase accuracy.
    History dependency does not require additional functional evaluation because previous state are stored during the simulation.
    These reduces the simulation runtime, while increasing the complexity in space of the algorithm.
    The length of the time series of states that needs to be stored  depends on the update formula and increases with the order.

      \paragraph{Algorithm}
      An implementation of the midpoint method can be found in \ref{algo:midpoint}.

      
\begin{algorithm}[H]
\DontPrintSemicolon
\SetKwComment{comment}{$\%$}{}
\SetKw{Int}{int}
\SetKw{To}{to}
\SetKw{Return}{return}
\SetKw{Not}{not}
\SetKw{Input}{Input}
\SetKw{Output}{Output}
\SetKw{False}{false}
\SetKw{True}{true}
\SetKwData{Item}{item}
\SetKwFunction{Min}{min}
\SetKwFunction{TitleFunction}{Midpoint method}

\caption{\protect\TitleFunction{}}
\label{algo:midpoint}

\Input: a system of ODEs $\frac{d[X]}{dt} = \vec{F}(t, [X])$ corresponding to a biochemical reaction system, the initial state $[X_0]$ of the system with species concentrations at time $0$, the simulation ending time $T_{\max}$ and the discretization stepsize $h$\;

\Output: a trajectory of the biochemical system expressed in terms of molecule concentrations with discretization stepsize $h$\;

$t = 0$\;
$[X_{old}] = [X_0]$\;
$[X] = [X_{old}] + \frac{h}{2}\left[\vec{F}(t,[X_{old}]) + \vec{F}(t+h, [X_{old}] + h\cdot\vec{F}(t, [X_{old}]))\right]$\;
$t = h$\;


\While{$t<T_{\max}$}{
	$[X_{new}] = [X_{old}] + 2h\cdot\vec{f}(t, [X])$\;
	$[X_{old}] = [X]$\;
	$[X] = [X_{new}]$\;
	$t = t+h$\;
}
\end{algorithm}


      \paragraph{Discussion}
      In order to preserve the order of accuracy of the MS algorithm, the one-step method used in the preliminary phase must have at least the same order of the MS method.

    \subsubsection{Simpson method}
    The Simpson method is a two-step implicit linear model which updates the system by:

    $$[X_{n+1}] = [X_{n-1}] + \frac{h}{3}[\vec{F}(t_{n-1}, [X_{n-1}]) + 4\vec{F}(t_n, [X_n]) + \vec{F}(t_{n+1}, [X_{n+1}])]$$

    \subsubsection{General linear multistep algorithm}
    An implementation of a generic $(s+1)$-step method that requires a preliminary phase where a one-step method is used to compute the first $s$ step of the simulation can be found in algorithm \ref{algo:s1-method}.

    \begin{algorithm}[H]
\DontPrintSemicolon
\SetKwComment{comment}{$\%$}{}
\SetKw{Int}{int}
\SetKw{To}{to}
\SetKw{Return}{return}
\SetKw{Not}{not}
\SetKw{Input}{Input}
\SetKw{Output}{Output}
\SetKw{False}{false}
\SetKw{True}{true}
\SetKwData{Item}{item}
\SetKwFunction{Min}{min}
\SetKwFunction{TitleFunction}{Linear $(s+1)$-step method}

\caption{\protect\TitleFunction{}}
\label{algo:midpoint}

\Input: a system of ODEs $\frac{d[X]}{dt} = \vec{F}(t, [X])$ corresponding to a biochemical reaction system, the initial state $[X_0]$ of the system with species concentrations at time $0$, the simulation ending time $T_{\max}$ and the discretization stepsize $h$ and the coefficient values $a_j, b_j$\;

\Output: a trajectory of the biochemical system expressed in terms of molecule concentrations with discretization stepsize $h$\;

$t = 0$\;
$[X_{old}] = [X_0]$\;
compute the first $s$ steps of the dynamics by a one-step numerical method of order at least equal to the implemented multistep method\;

\While{$t<T_{\max}$}{
	\If{$b_{-1}\neq 0$}{
		approximate $[X_{t+h}]$\;
	}
  $[X] = \sum\limits_{j=0}^sa_j[X_{n-j}] + h\sum\limits_{j=0}^sb_j\vec{F}(t_{n-j}, [X_{n-j}]) + hb_{-1}\vec{F}(t_{n+1}, [X_{n+1}])$\;
	$t = t+h$\;
}
\end{algorithm}


  \subsection{Adams methods}
  Adams methods are linear multistep methods that update the system state:

  $$[X_{n+1}] = [X_n] + h\sum\limits_{j=-1}^sb_j\vec{F}(t_{n-j}, [X_{n-j}])$$

  Where:

  \begin{multicols}{2}
    \begin{itemize}
      \item $n\ge s\ge 0$.
      \item $b_j$ are numbers that characterize the method.
        $b_{-1} = 0$ implies that the method is explicit and is called the Adams-Bashforth method, when it is implicit is the Adams-Moulton method.
    \end{itemize}
  \end{multicols}

  \subsection{BDF methods}
  BDF methods are implicit linear multistep methods that update the system by:

  $$[X_{n+1}] = \sum\limits_{j=0}^sa_j[X_{n-j}] + hb_{-1}\vec{F}(t_{n+1}, [X_{n+1}])$$

  Where:

  \begin{multicols}{2}
    \begin{itemize}
      \item $n\ge s\ge 0$.
      \item $a_j$ and $b_{-1}\neq 0$ fully characterize the system.
    \end{itemize}
  \end{multicols}

\section{Adaptive methods}
Adaptive numerical methods change the discretization stepsize at each simulation step so that $h$ remains close to the greatest value which keeps the global truncation error within a bound.
They do not set $h$ but a threshold for a maximum value of the global truncation error.
The output is a time series with time points not equally spaced.
They require the implementation of an error estimator.
Here one step methods are considered, particularly RK schemes.

  \subsection{Estimating the local truncation error}
  RK schemes are well suited to provide an efficient estimator of the local truncation error and to adapt the stepsize according to it.
  An a posteriori error estimator is used to estimate the local truncation error.
  This can be built in two ways:

  \begin{multicols}{2}
    \begin{itemize}
      \item By comparing the output of the same RK method with two different stepsizes $2h$ and $h$.
      \item By comparing the output of two RK methods of different order.
    \end{itemize}
  \end{multicols}

  In both cases the error estimator computes a posteriori, after the simulation step.
  The first case a considerable increase of computational cost is required, while the second do not by using two different RK methods with $s$ stages of order $p$ and $p+1$, which share the same set of values.

  \subsection{Runge-Kutta Fehlberg}
  The Runge-Kutta Fehlberg method of fourth order, or RK45 method updates the system as:

  $$[X_{n+1}] = [X_n] + \frac{25}{216}K_1+\frac{1408}{2565}K_3+\frac{2187}{4104}K_4-\frac{1}{5}K_5$$

  Coupled with a fifth order RK method:

  $$[\tilde{X}_{n+1}] = [X_n] + \frac{16}{135}K_1 + \frac{6656}{12825}K_3 + \frac{28561}{56430}K_4 =\frac{9}{50}K_5 + \frac{2}{55}K_6$$

  $K_1,\dots, K_6$ are shared between methods and are computed as:

  \begin{align*}
    K_1 &=h\vec{F}(t_n, [X_n])\\
    K_2 &=h\vec{F}\left(t_n+\frac{h}{4}, [X_n]+\frac{1}{4}K_1\right)\\
    K_3 &=h\vec{F}\left(t_n+\frac{3h}{8}, [X_n]+\frac{3}{32}K_1+\frac{9}{32}K_2\right)\\
    K_4 &=h\vec{F}\left(t_n+\frac{12h}{13}, [X_n]+\frac{1932}{2197}K_1-\frac{7200}{2197}K_2+\frac{7296}{2197}K_3\right)\\
    K_5 &=h\vec{F}\left(t_n+h, [X_n]+\frac{439}{216}K_1-8K_2+\frac{3680}{513}K_3-\frac{845}{4104}K_4\right)\\
    K_6 &=h\vec{F}\left(t_n+\frac{h}{2}, [X_n]-\frac{8}{27}K_1+2K_2-\frac{3544}{2565}K_3+\frac{1859}{4104}K_4-\frac{11}{40}K_5\right)
  \end{align*}

  The fourth order version is used to compute the dynamics of the system, while the fifth order one is used to estimate the local truncation error:

  $$\Delta_{n+1} = \frac{\biggr\vert [\tilde{X}_{n+1}]-[X_{n+1}]\biggr\vert}{h}$$

  The error estimate is compared to the error threshold $\epsilon_t$.
  If it is less the local truncation error is assumed smaller than the threshold and the state is accepted, in the other case the new state is rejected and evaluated again using a smaller $h$.
  In both cases $h$ is updates as:

  \begin{align*}
    h_{n+1} &=h_n\sigma\\
    \sigma &= \left(\frac{\epsilon_t}{2\Delta_{n+1}}\right)^{\frac{1}{4}}\approx 0.84\left(\frac{\epsilon_t}{\Delta_{n+1}}\right)^{\frac{1}{4}}
  \end{align*}

  This is derived from $\sigma = \left(\frac{\epsilon_t}{\Delta_{n+1}}\right)^{\frac{1}{p}}$, defining how to update $h$ for an adaptive one step method of order $p$.

    \subsubsection{Algorithm}
    An implementation of the RK45 method can be found at algorithm \ref{algo:rk45}.

    \begin{algorithm}[H]
\DontPrintSemicolon
\SetKwComment{comment}{$\%$}{}
\SetKw{Int}{int}
\SetKw{To}{to}
\SetKw{Return}{return}
\SetKw{Not}{not}
\SetKw{Input}{Input}
\SetKw{Output}{Output}
\SetKw{False}{false}
\SetKw{True}{true}
\SetKwData{Item}{item}
\SetKwFunction{Min}{min}
\SetKwFunction{TitleFunction}{RK45 algorithm}

\caption{\protect\TitleFunction{}}
\label{algo:rk45}

\Input: a system of ODEs $\frac{d[X]}{dt} = \vec{F}(t, [X])$ corresponding to a biochemical reaction system, the initial state $[X_0]$ of the system with species concentrations at time $0$, the simulation ending time $T_{\max}$, the initial discretization stepsize $h_0$ and a threshold for the maximum local truncation error $\epsilon_t$\;

\Output: a trajectory of the biochemical system expressed in terms of molecule concentrations with discretization stepsize $h_0$\;

$t = 0$\;
$[X] = [X_0]$\;
$h = h_0$\;


\While{$t<T_{\max}$}{
	$K_1 =h\vec{F}(t_n, [X_n])$\;
	$K_2 =h\vec{F}\left(t_n+\frac{h}{4}, [X_n]+\frac{1}{4}K_1\right)$\;
	$K_3 =h\vec{F}\left(t_n+\frac{3h}{8}, [X_n]+\frac{3}{32}K_1+\frac{9}{32}K_2\right)$\;
	$K_4 =h\vec{F}\left(t_n+\frac{12h}{13}, [X_n]+\frac{1932}{2197}K_1-\frac{7200}{2197}K_2+\frac{7296}{2197}K_3\right)$\;
	$K_5 =h\vec{F}\left(t_n+h, [X_n]+\frac{439}{216}K_1-8K_2+\frac{3680}{513}K_3-\frac{845}{4104}K_4\right)$\;
	$K_6 =h\vec{F}\left(t_n+\frac{h}{2}, [X_n]-\frac{8}{27}K_1+2K_2-\frac{3544}{2565}K_3+\frac{1859}{4104}K_4-\frac{11}{40}K_5\right)$\;
  $[X_{new}] = [X] + \frac{25}{216}K_1+\frac{1408}{2565}K_3+\frac{2187}{4104}K_4-\frac{1}{6}K_5$\;
  $[\tilde{X}_{new}] = [X] + \frac{16}{135}K_1 + \frac{6656}{12825}K_3 + \frac{28561}{56430}K_4 =\frac{9}{50}K_5 + \frac{2}{55}K_6$\;
	$\Delta = \frac{\biggr\vert [\tilde{X}_{new}]-[X_{new}]\biggr\vert}{h}$\;
	\If{$\Delta\le\epsilon_t$}{
		$[X] = [X_{new}]$\;
		$t = t+h$\;
	}
	$\sigma = 0.84\left(\frac{\epsilon_t}{\Delta}\right)^{\frac{1}{4}}$\;
	\If{$\sigma<0.1$}{
		$\sigma = 0.1$\;
	}
	\If{$\sigma > 4$}{
		$\sigma = 4$\;
	}
	$h = h\sigma$\;
}
\end{algorithm}


    \subsubsection{Discussion}
    Even though the computation of the simulation step is more demanding than the RK4 method the possibility of changing $h$ often decreases the simulation runtime.

  \subsection{Stiffness}
  An expected result of the adaptive method is that $h$ is small in region where the solution curves display large variation and large in region when the solution have a near zero slope.
  This doesn't always happen: the stepsize $h$ is forced to be small to an unacceptable level in region where solution curves are very small.
  This phenomenon is known as stiffness.
  A system of ODE is said to be stiff when a numerical solution of its initial value problem forces the numerical method to employ a discretization stepsize excessively small with respect to the smoothness of the exact solution.
  This is a property of the couple of ODE system and numerical scheme used to solve.
  It could appear or disappear changing the numerical scheme.
  It doesn't arise only in adaptive methods, but it is important because the algorithm may lose control on the update of $h$ with an impact on the runtime.
  In this cases the problem is solved by changing the numerical algorithm.
  Usually BDF methods are used to simulate a system of ODEs that exhibits stiffness with adaptive numerical simulation.

\section{Issues of deterministic simulation}
Regardless of the chosen numerical method the deterministic simulation provides only an approximation of the biochemical dynamics, this is often not a problem: the gain in simulation runtime compensates the loss in accuracy, especially for higher-order numerical methods.
However there are some specific conditions in which deterministic simulation fails to compute the real behaviour of the system.

  \subsection{Spatial homogeneity and continuum hypothesis}
  In particular spatial homogeneity and the continuum hypothesis are not satisfied for biological processes involving low numbered chemical species because the dynamics could be partially or totally driven by few stochastic events impossible to observe by considering averaged dynamics.
  Even when the two hypothesis are satisfied some properties are impossible to observed.

  \subsection{Multistability}
  One of them is multistability: this is never observed by an averaged dynamics of the system because the deterministic simulation computes the same averaged dynamics when starting from the same initial state.

  \subsection{Steady state}
  Another condition that lead to important differences arises when the biochemical system is simulated from a steady state, a solution of the set of ODEs when all derivative are set to $0$.
  This makes the averaged dynamics stationary.
  The asynchronous application of reaction firing in stochastic simulation could cause an exit from the equilibrium leading to non-stationary dynamics.
