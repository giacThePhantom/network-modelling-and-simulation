\graphicspath{{chapters/09/images}}
\chapter{Gradient methods}

\section{Introduction}
When the model needs to integrate to find an ODE solution, the output is points in that integrated function.
$f$ and $f'$ are not known of the optimization problem, so a numerical approximation is needed.
The idea of gradient methods is to set $f'=0$.
Gradient methods can be applied to both constraint and unconstrained models.
Integrating constraints the problem becomes:

$$\begin{cases}\max f(x) &  \\ g_i(x)=0, & i \in \mathcal{I} \\ x \in \mathbb{R}^N &\end{cases}$$

Where $\mathcal{I}= 1,...,m$.
The traditional way to solve this problem is to translate this system to another function.

  \subsection{Lagrangian function}
  The Lagrangian function is used to take into account the constraints.
  A Lagrangian function is defined as the function:

  $$L: \mathbb{R}^n \times \mathbb{R}^m \rightarrow \mathbb{R}$$

  Such that:

  $$L(x)= f(x) + \lambda g(x) = f(x) + \sum_{j=1}^m \lambda_j g_j(x)$$

  \subsection{Lagrangian Multipliers Theorem}
  If $x^*$ is a stationary point for a Lagrangian function, then $\exists \lambda^*$ such that $(x^*, \lambda^*)$ is a stationary point for $L$.
  This is a necessary condition that increases the dimensionality of the problem from $\mathbb{R}^N$ to $\mathbb{R}^N \times \mathbb{R}^m$, but it allows to search solutions using the stationary points.
  This can be generalized to $g_i(x)\leq 0$ constraints.
  Stationary points are not necessarily minima and maxima.
  This is enforced by checking through second derivations or evaluating the function in other points.

  \subsection{Definition of a gradient}
  Let $f: \mathbb{R}^N \rightarrow \mathbb{R}$ a differentiable function, the gradient of $f$ is:

  $$\nabla f: \mathbb{R}^N \rightarrow \mathbb{R}^N$$

  Where:

  $$\nabla f_i=\frac{\partial}{\partial x_i} f(x)_i$$

  And:

  $$\nabla f(x) =\left[\begin{array}{c}\frac{\partial}{\partial x_1} f(x) \\ \vdots \\ \frac{y}{\partial x_N} f(x)\end{array}\right]$$

  The objective is looking for points for which the derivative vanishes:

  $$x^* : \nabla f(x^*)=0$$

    \subsubsection{Some examples}
    Two functions used to test optimization algorithms are the Rosenbrock's function:

    $$f(x, y)=(1-x)^2+100\left(y-x^2\right)^2$$

    And the Eggholder function:

    $$f(x, y)=-(y+47) \sin \left(\sqrt{\biggr\vert \frac{x}{2}+(y+47)\biggr\vert}-\right. x \cdot \sin \left(\sqrt{| x-(y+47)|}\right)$$

    Which are used to test optimization algorithms.
    Solving analytically these two problems is hard, so a numerical solution has to be employed.

  \subsection{Limitations of gradient descent methods}
  One of the major limitations of these algorithms is that there is no guarantee to reach the global minimum.
  Furthermore it could be the case that variables cannot be optimized together, so there is a need to make a trade-off between them.
  Moreover gradient methods cannot be applied to stochastic simulations as their functions are not continuous.


  \begin{figure}[H]
    \centering
    \includegraphics[width=0.5\textwidth]{example.png}
    \caption{Blue = function, red = constraint}
    \label{fig:ex}
  \end{figure}

  Figure \ref{fig:ex} is an example of an objective function.
  If there is an equality constraint only the points meeting the boundary, in red, are to be considered.
  In an inequality constraint instead everything inside the red circle is considered.
  Generally constraints reduce the search space: the Lagrangian determines that the minimum point with some multipliers will give a solution of the Lagrangian.
  Finding the solution of the original condition.
  These could not be solutions of the original conditions, but they are ideal candidates that can be checked.
  To perform an evaluation of the distance the model should be integrated.
  This is computational expensive, so the measure of computation is the number of times the model has to be simulated per iteration.

\section{Gradient approximation with Taylor formula}
In most cases the gradient is unknown and it should be approximated using the Taylor formula.
Let $]a,b[ \in \mathbb{R}, x_0 \in ]a,b[$ and $f_i]a,b[ \rightarrow \mathbb{R}$ be differentiable $(n-1)$ times in $]a,b[$ and $f^{(n)}$ continuous in $x_0$.
Then let $x \in ]a,b[$:

$$f(x)=f\left(x_0\right)+f^{\prime}\left(x_0\right)\left(x-x_0\right)+f''(x_0)\frac{(x-x_0)^2}{2!}+ ...+ f^{(n)}(x_0)\frac{(x-x_0)^n}{n!} + R_n(x)$$

Such that:

$$\lim_{x \rightarrow x_0} \frac{R_n(x)}{(x-x_0)^n} =0$$

Focussing on the first terms:

$$f(x)=f\left(x_0\right)+f^{\prime}\left(x_0\right)\left(x-x_0\right)+R_2(x)=0$$

$$f^{\prime}\left(x_0\right)= \frac{f(x)-f(x_0)}{x-x_0}+\left(\frac{R_2(x)}{(x-x_0)}\right)\approx\frac{f(x)-f\left(x_0\right)}{x-x_0}+R_1(x)$$

This trick can be used for any $N>1$.
Now let $f: \mathbb{R}^N \rightarrow \mathbb{R}$ and $e_i=(0, \ldots, 0,1,0,\ldots,0)$.
Consider $x_1, x+\varepsilon \ell_i, x-\varepsilon \ell_i$ so that the gradient is being explored in one direction:

\begin{align*}
  f\left(x+\varepsilon e_i\right)&=f(x)+\varepsilon \frac{\partial f}{\partial x_i}(x)+\frac{1}{2} \varepsilon^2 \frac{\partial^2 f}{\partial x_i{ }^2}(x)+R_3(x) \\
  f\left(x-\varepsilon e_i\right)&=f(x)-\varepsilon \frac{\partial f}{\partial x_i}(x)+\frac{1}{2} \varepsilon^2 \frac{\partial^2 f}{\partial x_i{ }^2}(x)+R_3(x) \\
  \Rightarrow f\left(x+\varepsilon e_i\right)-f\left(x-\varepsilon e_i\right)&=+2 \varepsilon \frac{\partial f}{\partial x_i}(x)+R_3(x) \\
  \Rightarrow \frac{\partial f}{\partial x_i}(x)&=\frac{f\left(x+\varepsilon e_i\right)-f\left(x-\varepsilon e_i\right)}{2 \varepsilon}+R_2(x)
\end{align*}

This is an approximation of the first derivative with improved accuracy.
This applies to only one derivative, so it has to be performed at least twice as $2$ function evaluation are needed for each $i\rightarrow 2W$.
In order to obtain a decent gradient, a lot of computations are required, but these are fast as they have a low number of iterations.
The algorithm converges when $\Delta f = 0$.

  \subsection{Vanishing gradient}
  There might be points where the gradient vanishes which are not the final destination.
  Gradient methods may tend to overfit, but they are effective.
  The main issue is that since the gradient is approximated, it cannot be trusted everywhere.

\section{Line search}
The line search is a class of algorithms that follow a direction along the gradient.

  \subsection{Newton's direction}
  Considering Taylor's formula and letting $x_k$ be the starting point, $\alpha \in \mathbb{R}^+$ the step length and $p$ the direction such that ($x_k, p \in R^n$).
  For $n=1$:

  $$
  f(x_k+\alpha p)=f(x_k) + \alpha p f'(x_k) + \frac{\alpha^2p^2}{2} f''(x_k) + r(p^3)
  $$
  For simplicity set $\alpha=1$ and truncate the formula:
  $$
  f(x_n+p)=f(x_n) + p f'(x_n) + \frac{p^2}{2} f''(x_n) = m_k(p)
  $$

  Now instead of $f$ the simple polynomial $m_k(p)$ is minimized:

  $$\frac{d}{dp}m_k(p) = f'(x_n) + pf''(x_n) = 0$$

  From this equation the direction is:

  $$p = -\frac{f'(x_k)}{f''(x_k)}$$

  Or the Newton direction, which is the best direction to move the gradient.
  When $n>1$:

  $$p = -(\nabla^2 f(x_k))^{-1}\nabla f(x_k)$$

  Where:

  $$\nabla^2 f(x) = \left[\frac{\partial}{\partial x_i}\left(\frac{\partial}{\partial x_j}f(x)\right)\right]_{i,j}$$

  Is the Hessian matrix.
  Finding the inverse of a matrix is not straightforward, so next algorithm will try to approximate it.
  The second derivative progressively shrinks when the algorithm approaches convergence to reduce the step size.

  \subsubsection{Algorithm}
  An algorithm for line search with Newton's direction for function minimization can be found in \ref{algo:line-search}.

  \begin{algorithm}[H]
\DontPrintSemicolon
\SetKwComment{comment}{$\%$}{}
\SetKw{Int}{int}
\SetKw{To}{to}
\SetKw{Return}{return}
\SetKw{Not}{not}
\SetKw{Input}{Input}
\SetKw{Output}{Output}
\SetKw{False}{false}
\SetKw{True}{true}
\SetKwData{Item}{item}
\SetKwFunction{Min}{min}
\SetKwFunction{Partitioning}{partitioning}
\SetKwFunction{TitleFunction}{Line search}

\caption{\protect\TitleFunction{}}
\label{algo:line-search}

\Input: the function $f$ to be minimized and $x_k$ starting point\;

\Output: the minimized function\;

\While{$||\nabla f(x_k)||>\epsilon$}{
	$p_k = -(\nabla^2 f(x_k))^{-1}\nabla f(x_k)$\;
	select $\alpha_k$\;
	$x_{k+1} = x_k+\alpha_kp_k = x_k-\alpha_k(\nabla^2f(x_k))^{-1}\nabla f(x_k)$\;
	$k = k+1$\;
}

\end{algorithm}


  The stopping point is when the gradient is equal to zero, but there is a need to apply a threshold $\epsilon$ on it.
  Now at each iteration sometimes approximation is needed to compute the inverse of the matrix.
  Moreover all these gradients are computationally expensive to compute.
  Nevertheless this process of going in one direction is smart.
  The smartest part, or the computation of $p_k$ can be approximated to make the computation less expensive.


  \subsection{Quasi Newton's direction}
  To avoid the computation of $\nabla^2 f$, a different matrix $B_k$ is built, such that

  $$B_{k+1}\cdot (x_{k+1}-x_k) = \nabla f(x_{k+1})-\nabla f(x_k)$$

  The second derivate is being approximated, the gradient is exploited to prevent extra computation.
  The Hessian where will not be computed.
  So the quasi-Newton direction will be:

  $$p_{k+1} = -B_{k+1}^{-1}\nabla f_{k+1}$$

  The progressive shrinking of the second derivative indicates that the step needs to be reduced.
  The step $\alpha$ should be changed according to the Armijio condition.
  Consider in particular that each time that an operation on a matrix is performed precision is lost.

  \subsection{Steepest descent direction}
  In steepest descent the direction chosen is the one that reduces the gradient the most:

  $$p= -\alpha \frac{\nabla f}{||\nabla f||}$$

  The second derivative is being ignored and only the gradient computation is needed.
  This method converges more slowly with respect to Newton and Quasi Newton direction methods.

  \subsection{Selecting $\mathbf{\alpha}$}
  To determine the step length $\alpha$ consider:

  $$\phi(\alpha) = f(x_k + \alpha p_k)$$

  The objective is to define $\bar{\alpha}$ such that:

  $$\phi(\bar{\alpha}) = \min\limits_{\alpha>0}\phi(\alpha)$$

  This is another optimization problem, which could be avoided by selecting an $\alpha$ that satisfies other conditions

    \subsubsection{Armijio condition}
    The Armijio condition forces a selection of $\alpha$ such that:

    $$f(x_k+\alpha p_k) \leq f(x_k) + c_1\alpha \nabla f(x_n)^T p_k\qquad\qquad c_1 \in ]0,1[$$

    \subsubsection{Curvature condition}
    The curvature condition forces a selection of $\alpha$ such that:

    $$\nabla f(x_k+\alpha p_k)^Tp_k \ge c_2\nabla^Tf_kp_x\qquad\qquad c_2\in]\epsilon_1, 1[$$

  \subsection{Convergence of a method}
  The order of convergence of a method is a constant $\ell$, such that if $x^*$ is a solution, the limit:

  $$\lim_{k \rightarrow \infty  }\frac{||f(x_{k+1})-f(x^*)||}{||f(x_{k})-f(x^*)||^\ell} = L >0$$

  Exists.
  The new iteration is compared to the old one.
  The limit should go to $0$ and the exponent is the speed at which this happens: the higher $\ell$ the faster the numerator go to zero with the respect to the denominator, so the higher the exponent, the faster the convergence.
  For the methods discussed previously:

  \begin{multicols}{2}
    \begin{itemize}
      \item Steepest descent: $\ell=1$, linear convergence.
      \item Quasi-Newton: $\ell \in (1,2)$, superlinear convergence.
      \item Newton: $\ell=2$, quadratic convergence.
    \end{itemize}
  \end{multicols}


\section{Trust region}
Trust region is a class of algorithms that create an approximation of the problem and they solve it in a small trustable region.
The objective is to optimize a problem such that close to a starting point $x_k$ the model is close to the function.

  \subsection{Trust region steepest descent}
  Let now $f_k = f(x_k)$.
  The model can be approximated through a Taylor expansion:

  $$m(x_p+\alpha p)=f(x_k)+\alpha p^T \nabla f(x_k)+ \frac{1}{2} \alpha^2p^T B_k p$$

  Where $B_k$ can be the Hessian matrix or its approximation.
  This problem can be solved in a region such that:

  $$||p||_2<\delta_k\qquad\qquad \delta_k > 0$$

  Now assume for simplicity that $\alpha = 1$ and that $B_k = 0$.
  The model of the function will be now:

  $$m_k(p) = f_k + p^T\nabla f_k$$

  To minimize this, since $f_k$ is a constant there is a need to work only on the second term:

  $$p^T\nabla f_k = ||p||\cdot||\nabla f_k||\cdot \cos\theta$$

  The minimum is reached when $\cos\theta = -1$ and $||p|| = \delta_k$, so that $\bar{p}$:

  $$\bar{p}^T\nabla f_k = -\delta_k||\nabla f_k||\Rightarrow \bar{p} = -\delta_k\frac{\nabla f_k}{||\nabla f_k||}$$

  This result is exactly the equation from steepest descent.
  A condition on the region with $\delta_k$ is being applied.
  This direction and the whole approach is called trust region steepest descent.
  The same idea could be followed by applying Newton or Quasi-Newton.

  \subsection{Evaluating the trust regions}
  To evaluate the trust regions, the actual reduction is defined as:

  $$\rho_k = \frac{f(x_k)-f(x_k+p)}{m(x_k)-m(x_k+p)}$$

  By definition $m(x_k+p) \leq m(x_k)$, so the denominator is always $>0$.

  \begin{multicols}{2}
    \begin{itemize}
      \item If $\rho_k < 0$, reject $p$, as the real problem is not being improved.
        $\delta_k$ should be reduced, typically by $\frac{1}{4} \delta_k$.
      \item The closer $\rho_k$ is to $1$ the closer the model $m_k$ to $f$, so bigger steps can be taken increasing $\delta_k$.
    \end{itemize}
  \end{multicols}

  The value of $\delta$ can be tuned according to the needs of the problem.
  The approach is similar to  the RK method.
  A grey area  between $0$ and $1$ is found, so a threshold is defined such that $\rho_k < \eta$ and $\rho_k > \eta$.


  \subsection{Trust region algorithm}
  An implementation of the trust region method is outlined in algorithm \ref{algo:trust-region}.

  \begin{algorithm}[H]
\DontPrintSemicolon
\SetKwComment{comment}{$\%$}{}
\SetKw{Int}{int}
\SetKw{To}{to}
\SetKw{Return}{return}
\SetKw{Not}{not}
\SetKw{Input}{Input}
\SetKw{Output}{Output}
\SetKw{False}{false}
\SetKw{True}{true}
\SetKwData{Item}{item}
\SetKwFunction{Min}{min}
\SetKwFunction{Partitioning}{partitioning}
\SetKwFunction{TitleFunction}{Trust region}

\caption{\protect\TitleFunction{}}
\label{algo:trust-region}

\Input: the function $f$ to be minimized and $x_k$ starting point, $\hat{\delta}$ the maximum accepted region, the starting $\delta\in ]0, \hat{\delta}[$ and $\eta\in\left[0, \frac{1}{4}\right]$ the minimum actual reduction for which the direction is accepted.\;

\Output: the minimized function\;

\Repeat{$||\nabla f(x_k)||<\epsilon$}{
	obtain $p_k$ such that $p_k=\arg\min\limits_{\substack{p\in\mathbb{R}^n\\||p||\le\delta_k}}m(x_k+p)$\;
	compute $\rho_k$\;
	\If{$\rho_k<\frac{1}{4}$}{
		$\delta_{k+1} = \frac{1}{4}\delta_k$\;
	}
	\ElseIf{$\rho_k > \frac{3}{4}\land||p_k||=\delta_k$}{
		$\delta_{k+1} = \min(2\delta_k, \hat{\delta})$\;
	}
	\Else{
		$\delta_{k+1} = \delta_k$\;
	}
	\If{$\rho_k >\eta$}{
		$x_{k+1} = x_k +p_k$\;
	}
	\Else{
		$x_{k+1} = x_k$\;
	}
}

\end{algorithm}


  The stopping point is when the gradient is sufficiently small.
  The focus is on computing $p_k$, then $\rho_k$ is evaluated to adjust the parameters.
